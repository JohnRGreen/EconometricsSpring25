
% Inbuilt themes in beamer
\documentclass[aspectratio=169]{beamer}
\usepackage{graphicx}
\usepackage{mathrsfs}
% Theme choice:
\usetheme{CambridgeUS}

% Title page details: 
\title{Econometrics Discussion Section 2: Omitted Variable Bias} 
\author{John Green}
\date{Spring 2025}

\begin{document}

% Title page
\begin{frame}
    \titlepage 
\end{frame}

\begin{frame}
    \frametitle{Mis-specified model}
    \begin{itemize}
        \item Suppose the true model is:
        $$
            Y = \beta_0 + \beta_1 X + \beta_2 Z + \epsilon
        $$
        \item But we estimate the mis-specified model:
        $$
            Y = \beta_0 + \beta_1 X + u
        $$
        where $u = \beta_2 Z + \epsilon$
        \item We already know that our estimator $\hat{\beta}_1^{OLS}$ for $\beta_1$ will be biased if:
    \end{itemize}
\end{frame}

\begin{frame}
    \frametitle{Mis-specified model}
    \begin{itemize}
        \item Suppose the true model is:
        $$
            Y = \beta_0 + \beta_1 X + \beta_2 Z + \epsilon
        $$
        \item But we estimate the mis-specified model:
        $$
            Y = \beta_0 + \beta_1 X + u
        $$
        where $u = \beta_2 Z + \epsilon$
        \item We already know that our estimator $\hat{\beta}_1^{OLS}$ for $\beta_1$ will be biased if \textit{$X$ and $Z$ are correlated}: $cov(X,Z) \neq 0 $.
    \end{itemize}
\end{frame}

\begin{frame}
    \frametitle{Mis-specified model}
    \begin{itemize}
        \item This is called \textit{omitted variable bias} (OVB): by failing to include a variable in our model, we get a biased (and inconsistent) estimate of $\beta_1$
        \item The corollary is that we do not have OVB if $X$ and $Z$ are Uncorrelated
        \begin{itemize}
            \item Important, since there are \textit{always} going to be omitted variables in the error
        \end{itemize}
    \end{itemize}
\end{frame}

\begin{frame}
    \frametitle{Analyzing the bias}
    \begin{itemize}
        \item We can actually work out the bias in $\hat{\beta}_1^{OLS}$ when we omit $Z$:
        $$
            \hat{\beta}_1^{OLS} \rightarrow_p \beta_1 + \dfrac{\sigma_u}{\sigma_X} \times \rho_{X,u}
        $$
        \item With a careful analysis, we can start to understand the magnitude and direction of the bias
        \item Focus on first term: $\dfrac{\sigma_u}{\sigma_X}$
        \item How does bias change with $\sigma_u$ and $\sigma_X$?
    \end{itemize}
\end{frame}

\begin{frame}
    \frametitle{Analyzing the bias}
    \begin{itemize}
        \item We can actually work out the bias in $\hat{\beta}_1^{OLS}$ when we omit $Z$:
        $$
            \hat{\beta}_1^{OLS} \rightarrow_p \beta_1 + \dfrac{\sigma_u}{\sigma_X} \times \rho_{X,u}
        $$
        \item How does bias change with $\sigma_u$ and $\sigma_X$?
        \begin{itemize}
            \item If $\sigma_u$ is ``large'' relative to $\sigma_X$, then the bias is large
            \item Or, if $\sigma_X$ is ``small'' relative to $\sigma_u$, then the bias is large (same thing)
        \end{itemize}
        \item important to keep in mind what $u$ is: \textbf{not} just $Z$, but $\beta_2 Z$.
        \item What about $\rho_{X,u}$?
    \end{itemize}
\end{frame}

\begin{frame}
    \frametitle{Analyzing the bias}
    \begin{itemize}
        \item We can actually work out the bias in $\hat{\beta}_1^{OLS}$ when we omit $Z$:
        $$
            \hat{\beta}_1^{OLS} \rightarrow_p \beta_1 + \dfrac{\sigma_u}{\sigma_X} \times \rho_{X,u}
        $$
        \item important to keep in mind what $u$ is: \textbf{not} just $Z$, but $\beta_2 Z$.
        \item What about $\rho_{X,u}$?
        \item Will depend on two things: 
        \begin{itemize}
            \item Relationship between $Z$ and $Y$: $\beta_2$ 
            \item Relationship between $Z$ and $X$: $cov(X,Z)$ 
        \end{itemize}
        \item If they move in different directions, then $\rho_{X,u} < 0$ and the bias is negative; same direction, then $\rho_{X,u} > 0$ and the bias is positive
    \end{itemize}
\end{frame}

\begin{frame}
    \frametitle{Example \#1}
    \begin{itemize}
        \item Regress wages on years of education:
        $$ 
            w_i = \beta_0 + \beta_1 Educ_i + u_i
        $$
        \item What might be omitted?
    \end{itemize}
\end{frame}


\begin{frame}
    \frametitle{Example \#1}
    \begin{itemize}
        \item Regress wages on years of education:
        $$ 
            w_i = \beta_0 + \beta_1 Educ_i + u_i
        $$
        \item Omitted variable is field of study: think about becoming a doctor vs. becoming a nurse, or paralegal vs. lawyer
        \item So ``true'' model is $w_i = \beta_0 + \beta_1 Educ_i + \beta_2 major_i + e_i$
        \item What direction is bias?
    \end{itemize}
\end{frame}

\begin{frame}
    \frametitle{Example \#1}
    \begin{itemize}
        \item Regress wages on years of education:
        $$ 
            w_i = \beta_0 + \beta_1 Educ_i + u_i
        $$
        \item Omitted variable is average time to degree in type of degree: think about becoming a doctor vs. becoming a nurse, or paralegal vs. lawyer
        \item So ``true'' model is $w_i = \beta_0 + \beta_1 Educ_i + \beta_2 degree_i + e_i$
        \item $\beta_2 > 0$ (longer time to degree $\rightarrow$ degree brings higher wage)
        \item $cov(Educ_i, degree_i) > 0 $ (longer time to degree $\rightarrow$ more years of education)
        \item So bias should be...
    \end{itemize}
\end{frame}

\begin{frame}
    \frametitle{Example \#1}
    \begin{itemize}
        \item Regress wages on years of education:
        $$ 
            w_i = \beta_0 + \beta_1 Educ_i + u_i
        $$
        \item ``True'' model is $w_i = \beta_0 + \beta_1 Educ_i + \beta_2 degree_i + e_i$
        \item $\beta_2 > 0$ (longer time to degree $\rightarrow$ degree brings higher wage)
        \item $cov(Educ_i, degree_i) > 0 $ (longer time to degree $\rightarrow$ more years of education)
        \item So bias should be positive: by ignoring degree type, years of education will appear \textbf{more} important than they really are
    \end{itemize}
\end{frame}


\begin{frame}
    \frametitle{Example \#2}
    \begin{itemize}
        \item Regress wages on years of experience:
        $$ 
            w_i = \beta_0 + \beta_1 exper_i + u_i
        $$
        \item What might be omitted?
    \end{itemize}
\end{frame}


\begin{frame}
    \frametitle{Example \#2}
    \begin{itemize}
        \item Regress wages on years of experience:
        $$ 
            w_i = \beta_0 + \beta_1 exper_i + u_i
        $$
        \item Omitted variable is health; poor health leads to lower wages, and also keeps you out of the labor force
        \item So ``true'' model is $w_i = \beta_0 + \beta_1 Educ_i + \beta_2 health_i + e_i$
        \item What direction is bias?
    \end{itemize}
\end{frame}

\begin{frame}
    \frametitle{Example \#2}
    \begin{itemize}
        \item Regress wages on years of experience:
        $$ 
            w_i = \beta_0 + \beta_1 exper_i + u_i
        $$
        \item Omitted variable is health; poor health leads to lower wages, and also keeps you out of the labor force
        \item ``True'' model is $w_i = \beta_0 + \beta_1 Educ_i + \beta_2 health_i + e_i$
        \item $\beta_2 < 0$ (poor health $\rightarrow$ lower wages)
        \item $cov(exper_i, health_i) < 0 $ (poor health $\rightarrow$ less experience)
        \item So again, bias will be positive; by ignoring health, years of experience will appear \textbf{more} important than they really are
    \end{itemize}
\end{frame}

\begin{frame}
    \frametitle{Example \#3}
    \begin{itemize}
        \item Regress wages on GPA:
        $$ 
            w_i = \beta_0 + \beta_1 GPA_i + u_i
        $$
        \item Omitted variables might be...
    \end{itemize}
\end{frame}

\begin{frame}
    \frametitle{Example \#3}
    \begin{itemize}
        \item Regress wages on GPA:
        $$ 
            w_i = \beta_0 + \beta_1 GPA_i + u_i
        $$
        \item Omitted variable could be difficulty of major, $d_i$
        \item ``True'' model is $w_i = \beta_0 + \beta_1 GPA_i + \beta_2 d_i + e_i$
        \item What direction is bias?
    \end{itemize}
\end{frame}

\begin{frame}
    \frametitle{Example \#3}
    \begin{itemize}
        \item Regress wages on GPA:
        $$ 
            w_i = \beta_0 + \beta_1 GPA_i + u_i
        $$
        \item Omitted variable could be difficulty of major, $d_i$
        \item ``True'' model is $w_i = \beta_0 + \beta_1 GPA_i + \beta_2 d_i + e_i$
        \item Difficulty leads to lower GPA ($cov(GPA_i, d_i) < 0$), but higher wages ($\beta_2 > 0$)
        \item So direction of bias is...
    \end{itemize}
\end{frame}

\begin{frame}
    \frametitle{Example \#3}
    \begin{itemize}
        \item Regress wages on GPA:
        $$ 
            w_i = \beta_0 + \beta_1 GPA_i + u_i
        $$
        \item Omitted variable could be difficulty of major, $d_i$
        \item ``True'' model is $w_i = \beta_0 + \beta_1 GPA_i + \beta_2 d_i + e_i$
        \item Difficulty leads to lower GPA ($cov(GPA_i, d_i) < 0$), but higher wages ($\beta_2 > 0$)
        \item So direction of bias is negative: GPA will appear \textit{less} important than it really is, since we are ignoring the confounding effect of major.
    \end{itemize}
\end{frame}

\begin{frame}
    \frametitle{Example \#4}
    \begin{itemize}
        \item Regress wages on GPA:
        $$ 
            w_i = \beta_0 + \beta_1 GPA_i + u_i
        $$
        \item Another omitted variable could be use of AI
    \end{itemize}
\end{frame}

\begin{frame}
    \frametitle{Example \#4}
    \begin{itemize}
        \item Regress wages on GPA:
        $$ 
            w_i = \beta_0 + \beta_1 GPA_i + u_i
        $$
        \item Another omitted variable could be use of AI
        \item ``True'' model is $w_i = \beta_0 + \beta_1 GPA_i + \beta_2 AI_i + e_i$
        \item What direction is bias?
    \end{itemize}
\end{frame}

\begin{frame}
    \frametitle{Example \#4}
    \begin{itemize}
        \item Regress wages on GPA:
        $$ 
            w_i = \beta_0 + \beta_1 GPA_i + u_i
        $$
        \item Another omitted variable could be use of AI
        \item ``True'' model is $w_i = \beta_0 + \beta_1 GPA_i + \beta_2 AI_i + e_i$
        \item Using AI leads to higher GPA ($cov(GPA_i, AI_i) > 0$), but lower wages ($\beta_2 < 0$) since not learning the skills
        \item So direction of bias is...
    \end{itemize}
\end{frame}

\begin{frame}
    \frametitle{Example \#4}
    \begin{itemize}
        \item Regress wages on GPA:
        $$ 
            w_i = \beta_0 + \beta_1 GPA_i + u_i
        $$
        \item Another omitted variable could be use of AI
        \item ``True'' model is $w_i = \beta_0 + \beta_1 GPA_i + \beta_2 AI_i + e_i$
        \item Using AI leads to higher GPA ($cov(GPA_i, AI_i) > 0$), but lower wages ($\beta_2 < 0$) since not learning the skills
        \item So direction of bias is negative: GPA will appear \textit{less} important than it really is, since we are ignoring the confounding effect of AI.
    \end{itemize}
\end{frame}

\begin{frame}
    \frametitle{Solutions}
    \begin{itemize}
        \item This seems very troubling; lots of variables out there to omit
        \item What can we do?
    \end{itemize}
\end{frame}

\begin{frame}
    \frametitle{Solutions}
    \begin{itemize}
        \item This seems very troubling; lots of variables out there to omit
        \item What can we do?
        \item Include it in the regression! (if you can)
        \item Use some control variable (instrument) which may proxy for the omitted variable
        \item Research design: diff-in-diff, RCT
        \item Advanced techniques: fixed effects, matching methods, regression discontinuity or kink; worst case scenario, try to bound the extent of bias
    \end{itemize}
\end{frame}

\end{document}